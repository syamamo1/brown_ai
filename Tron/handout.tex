\documentclass{article}
\usepackage{amssymb}
\usepackage{amsmath}
\usepackage[utf8]{inputenc}
\usepackage{hyperref}
\hypersetup{
    colorlinks=true,
    linkcolor=blue,
    filecolor=magenta,
    urlcolor=blue,
}
\usepackage{fullpage}
\newcommand{\amy}[1]{{\color{blue}[Amy: #1]}}
\newcommand{\george}[1]{{\color{red}[George: #1]}}
\newcommand{\jinwoo}[1]{{\color{magenta}[Jinwoo: #1]}}

\title{CSCI1410 Fall 2020 \\
Final Project: Tron-141}

\date{Project Due Monday, December 14 at 11:59pm ET \\ [1ex]
  \textbf{December 14 is a hard deadline. \\
    No late projects will be graded, and \\
    no late days can be used for the final project. \\}}

\begin{document}

\maketitle


\section{Important Dates}

\begin{center}
\begin{tabular}{|l||l|l|}
\hline    
Milestones & Date & Time \\
\hline    
\hline    
Partner Form Due                 & 11/25 & 11:59am ET \\
Warm-up Bot \& Writeup Due       & 12/7  & 5:59pm ET \\ 
Mentor TA Meeting Deadline       & 12/7  & 5:59pm ET \\ 
Tournament Begins                & 12/7  & 11:59pm ET \\
Final Bot \& Writeup Due         & 12/14 & 11:59pm ET \\
Tournament Ends                  & 12/15 & 11:59pm ET \\
\hline    
\end{tabular}
\end{center}

\textbf{Note:} If you are taking this course as a Capstone, you must work individually.
Otherwise, you are encouraged---but not required---to work in pairs.
You may choose your own partner, or ask us for help finding a partner.
To request our help, it is best to communicate with us via Piazza.

Either way, \textbf{you are required to submit a Google form informing
  us of your plans by 11/25 at 11:59am ET.}  Do not miss this
deadline.  We cannot assign you a Mentor TA until you submit this
form, and the sooner you are assigned a Mentor TA, the sooner you can
get started in earnest on the project.


\section{Goals}
The goal of this final project is to help you synthesize all the AI
knowledge you have garnered over the course of the semester.  The
project entails building an AI bot to play a grid game (i.e., a game
played on a discrete grid).  To build an effective bot for this task
will require that you employ multiple techniques, ranging from
adversarial search to machine learning (i.e., function approximation)
to reinforcement learning to multi-armed bandits.  The best bots will
utilize all of these techniques and more.


\section{The Game}
The game of Tron derives from a Disney movie of the same name.  In the
game, two motorcycles (agents) drive around a wall-enclosed grid at a
constant speed.  The motorcycles can continue in the direction they
are going, turn right, or turn left.
%(making $90^{\circ}$ turns).
Whenever they exit a cell, they leave behind an impenetrable barrier,
which makes that cell uninhabitable.  Eventually, one of the two
agents has no choice but to crash into a wall or a barrier.  At that
point, the game ends with the agent who crashed as the loser.

As studied in the AI literature
(e.g.,~\cite{teuling12,lanctot13,samothrakis10}, Tron is a two-player,
simultaneous-move, zero-sum game.  Tron-141, however, which is the
game we have designed as the final project for CSCI 1410, is a
sequential-move game, with the two agents moving in turn.  Many Tron
agents built for the simultaneous-move version use heuristic
strategies that incorporate aspects of adversarial search (e.g.,
$\alpha\beta$ pruning), making them readily applicable to
alternating-move games.  We have eliminated some of the original
game's complexity by changing the rules in an arguable unnatural way
such that only one motorcycle can drive at a time.


\subsection{The Rules}
Tron-141 is a two-player, alternating-move, zero-sum game played on a
walled-in rectangular grid (i.e., the board), in which players take
turns moving straight ahead, left, or right,
%in one of four directions---Up, Down, Left, or Right
%\amy{aren't there only three directions, since if you move backwards you die immediately?}
leaving behind an impenetrable barrier.
%
A player loses by colliding with a barrier or a wall.

Below are two example 7x7 game boards.  The one on the left is the
initial board (also called a \textbf{map}), and the one on the right
is the same board after Player 1 has moved down and Player 2 has moved
up.

\smallskip
\begin{verbatim}
                 #######                       #######
                 #1    #                       #x    #
                 #     #                       #1    #
                 #     #                       #     #
                 #     #                       #    2#
                 #    2#                       #    x#
                 #######                       #######
\end{verbatim}
\smallskip

\noindent
The numbers 1 and 2 denote the current locations of Players 1 and 2,
respectively; the \verb|#| symbols denote permanent walls; and the
\verb|x| symbols represent the barriers that the players have left
behind.

Below are two example 13x13 game boards.  These boards are initialized
with additional walls, beyond those enclosing the board.
Note that the players initial positions can vary.

\smallskip
\begin{verbatim}
             ###############               ###############
             #             #               #             #
             #  ##         #               #    1        #
             #  ## 1       #               #    #####    #
             #     ###     #               #    #   #    #
             #     ###     #               #    #   #    #
             #       2 ##  #               #    #####    #
             #         ##  #               #        2    #
             #             #               #             #
             ###############               ###############
\end{verbatim}
\smallskip

\textbf{Note:} Tron-141, in its full generality, also includes
powerups.  These powerups are present in the code, and explained in
Appendix~\ref{app:powerups}.  While it may be fun for you to tailor
your AI bot to powerups, this project is sufficiently rich without
them, so you need not pay them any mind.

\textbf{Another Note:} To get a feel for how the game works, we
recommend you run through a few example games.  To do so, run
\verb|gamerunner.py| without any command-line arguments.  This code
will execute a game between two bots who choose their moves
randomly, and print the stream of boards to your terminal.


\subsection{Time Limit}
Each player must move within \textbf{1 second}.
If a player does not move within this time frame,
the simulator moves them Up.
%
Furthermore, we cannot accept multithreaded bots, because code that
inadvertently is not thread safe could jeopardize our class
tournaments.


\subsection{Evaluation}
We will evaluate the success of your AI bot in several ways.  We will
evaluate your core ideas on the basis of your writeup.  We will also
test your bot's performance against various TA bots. 
%Your grade will be a combination of your successes along
%all of these dimensions.
%two of which use $\alpha\beta$-pruning with evaluation heuristics of two different strengths.

Finally, during finals week and beyond, we will run a tournament,
where all the Tron-141 bots developed this year will compete against
one another.  Your grade will be a combination of your successes along
all of these dimensions.


\section{Approaches}

About a decade ago, solving Tron was an ongoing research challenge.
Here are links to a few papers chock full of ideas that should be
useful as you work on this project:

\begin{itemize}
\item \href{https://project.dke.maastrichtuniversity.nl/games/files/bsc/Kang_Bsc-paper.pdf}{Endgame Detection in Tron}
  \\ [-4ex]
  
\item \href{https://ieeexplore.ieee.org/document/5593331}{A UCT Agent for Tron: Initial investigations}
  \\ [-4ex]

\item \href{https://dke.maastrichtuniversity.nl/m.winands/documents/Tronpaper.pdf}{Monte-Carlo Tree Search for the Simultaneous-Move Game Tron}
  \\ [-4ex]

\item \href{http://mlanctot.info/files/papers/sm-tron-bnaic2013.pdf}{Monte Carlo Tree Search for Simultaneous-Move Games: A Case Study in the Game of Tron}%
\footnote{The last two papers have overlapping authors, and hence, quite likely, overlapping ideas.}
  \\ [-4ex]

\item Google even ran a Tron competition; one competitor's post mortem is posted
\href{https://www.a1k0n.net/2010/03/04/google-ai-postmortem.html}{here}.
  \\ [-4ex]
\end{itemize}


\subsection{Adversarial Search}
The most basic approach to building an AI bot to play a game is to
simply hard-code some reasonable heuristic behavior.  A popular
heuristic for Tron is the \emph{wall-following\/} heuristic, which
favors moving along the walls.  This heuristic is implemented in one
of the weaker TA bots, but note that wall-following is not a terrible
idea if your opponent is also a wall-follower,
%\footnote{Not to be confused with a ``wall-flower,'' a similar concept.}
especially if the games were scored according to how long they last.
%
An alternative to wall-following is the \emph{path-planning\/}
heuristic, in which a player searches for a move that would afford it
the longest continuation: i.e., the longest available path following said move.

A more principled approach to solving any two-player,
alternating-move, zero-sum game is to use the minimax algorithm.  A
more efficient yet equally principled approach (since it is provably
equivalent to minimax in terms of the solutions its finds) is to use
$\alpha\beta$-pruning, which can prune nodes in the search tree.
%
But even $\alpha\beta$-pruning is not efficient enough to solve Tron,
without artificially inhibiting the depth of its search.
Consequently, heuristic evaluation functions that incorporate domain
knowledge are required, much like they were for Connect Four, in the
Adversarial Search homework assignment.

The aforementioned papers outline several heuristic evaluation
functions, including those implemented in the TA bots.  Most, if not
all, the heuristics employ some form of ``space estimation,'' because
it is advantageous for a player to be surrounded by free space, rather
than walls and barriers.  There are various ways to estimate space.
One of the more naive approaches simply counts the number of
contiguous cells accessible to a player,%
\footnote{\texttt{TA-Bot1} uses $\alpha\beta$-pruning with a heuristic that
  estimates a player's free space.}  while a more sophisticated
variant counts the length of the longest paths in this region, since
it is not possible to traverse all free cells.  But neither of these
approaches pay any mind to the opponent---who is vying for the very
same space!  So what makes more sense in this game is to compute a
(naive or sophisticated) space estimate for \emph{both\/} players, and
then to define a heuristic evaluation function that combines these
estimates in some way (e.g., take their difference, their ratio,
etc.), as it is these estimates combined that contains the most
information about which player is \emph{en route\/} to winning the
game.

An alternative heuristic evaluation function which bakes in
consideration of the opponent is described (with pictures!) in
\href{https://project.dke.maastrichtuniversity.nl/games/files/bsc/Kang_Bsc-paper.pdf}{Endgame Detection in Tron}.
%
This approach labels each cell as closer to either Player 1 or 2, as
measured via Manhattan distance.  These labelled regions are called
\emph{Voronoi regions}, while the cells that are equidistant to both
players are called the ``battlefront.''  As above, the heuristic value
is then the difference in the sizes of the two players' Voronoi
regions, thereby predicting the winner to be whichever player's
Voronoi region is larger.  This heuristic evaluation function is
sensible because the player with the larger Voronoi region can win the
game by moving directly to the battlefront and then proceeding to
cordon off their Voronoi region.  \texttt{TA-Bot2}, the best of the TA
bots, uses $\alpha\beta$-pruning with this Voronoi heuristic.


\subsection{Reinforcement Learning}
As we discussed in class, state-of-the-art methods for AI game
playing, like AlphaGo~\cite{silver16}, employ reinforcement learning
(RL) methods.  Recall that RL is applicable in Markov decision
processes (MDPs).  Hence, to employ RL in a game requires that we view
the game as an MDP.  This reduction is almost entirely
straightforward: the states in the MDP are the board configurations
(i.e., the nodes in an adversarial tree search); the actions in the
MDP are the players' available moves at each board configuration; and
the rewards at the terminal states indicate who the winner is.  Only
the transition probabilities are potentially ill-defined.  But given
an opponent strategy (e.g., a wall-following agent), their behavior
defines the transition probabilities.  Moreover, so long as that
opponent's strategy is stationary---meaning the distribution over
actions it employs depends only on the state but not on time---the
Markov property is satisfied.

\textbf{N.B.} Whereas an adversarial search tree models \emph{both\/}
players, so that nodes are labelled with players' identities
indicating whose turn it is to move, the aforementioned MDP models
only one player, so that every state corresponds to just one player's
actions.  The other player's actions are folded into the transition
probabilities; they are not modelled explicitly.

Given this reduction, it is straightforward in principle to learn to
play the game of Tron (or even the game of Go) using RL.  Well, not so
fast!  Just as $\alpha\beta$-pruning is intractable in Tron (and Go)
if it is not depth-limited (i.e., it is impossible to visit all nodes
in the search tree), it is likewise impossible to evaluate all states
in this MDP.  However, it is not intractable to evaluate a few states.
Hence, the way to use RL in game-playing is to combine it with
supervised learning (\emph{a.k.a.\/} function approximation), so that
we can learn the values of a few states, and then generalize those
values across many states.  As usual, we can represent states (i.e.,
board configurations) in terms of their features, use non-linear basis
functions to transform those features into a richer ``derived''
feature space, and then regress, either in a batch fashion using least
squares, or incrementally, using stochastic gradient descent, to learn
a value function for the MDP.%
\footnote{A point of clarification about our nomenclature: We use the
  term ``heuristic evaluation function'' when discussing a heuristic
  that is applied to a node/state to estimate its value (i.e., which
  player will win).  We then use the term ``value function'' to denote
  the values are all nodes/states that have been backed up (in the
  sense of minimax or, equivalently Bellman, in our MDP formulation of
  the adversarial search tree) throughout the game tree/MDP.}

In principle, if we could visit a small but representative sample of
the game states, it is conceivable that we could create a data set
consisting of precisely those representative game states and their
estimated values, from which we could generalize effectively to all
the game states.  But who is to say a small, representative sample
even exists?  Here is another, related, idea.  What if instead of
trying to learn the value function, we instead tried to learn an
optimal policy outright, via an algorithm like policy iteration?
Just as we are unlikely to be able to visit enough states to learn the
value function exactly, it is also likely impossible to learn an
optimal policy at \emph{all\/} states.  Still, perhaps it is possible
to learn an optimal policy at least at the \emph{relevant\/} states:
i.e., those which are encountered often.  This idea of using policy
iteration to try to learn optimal actions at relevant states dates
back at least to TD-Gammon~\cite{tdgammon}.  

\if 0
One popular approach, dating back to at least TD-Gammon, is to
iteratively learn a value function, using the current value function
to guide the search for a representative sample during each iteration.
%
%The value function is initialized somehow, and then iteratively improved,
%as more and more states and their corresponding values are sampled.
Here is an outline of this learning procedure, which we dub RL/SL:

\begin{table}[h!]
\begin{enumerate}
\item Initialize the value function.
  
\item For $K$ iterations (i.e., learning epochs):

\begin{enumerate}
\item Use the current value function to generate sample data (state-value pairs).
  
\item Use regression to learn a new value function from the data.
\end{enumerate}

\end{enumerate}
\caption{RL/SL Learning Procedure}
\label{alg:rl-sl}
\end{table}
\fi

\begin{table}[ht!]
\begin{enumerate}
\item Initialize $\pi$ to a prior policy.
 
\item For $K$ iterations (i.e., learning epochs):

\begin{enumerate}
\item Use $\pi$ to walk the tree, generating sample data.
  %(state-value pairs).
  %\\ The simplest way to do this is simply to create a data point for every state visited as follows:
  %(state, sample average value of all the next states according to the current value function)

\item Use your favorite regression method to learn $V^\pi$ from the data.

\item Construct a new policy $\pi$ from the old policy $\pi$ and the value function $V^{\pi}$.
\end{enumerate}

\end{enumerate}
\caption{Policy Iteration: Alternating Policy Evaluation with Policy Improvement}
\label{alg:policy-iteration}
\end{table}

Recall that policy iteration alternates between policy evaluation and
policy improvement.  (See Table~\ref{alg:policy-iteration}.)  The key
question we face when learning to play games with large state spaces
is: how do we to walk the tree (Step 2(a)) to generate
\emph{relevant\/} sample data?
%(i.e., states and their corresponding values)?
That is, how do we use the information contained in the current value
function---the only information we have---to seed these walks, to learn
about relevant states?
 
A popular way to tackle this problem is via \emph{Monte Carlo Tree
  Search} (MCTS), which uses repeated Monte Carlo sampling from the
root to build a subtree of the (intractable) game tree.  The data then
comprise all the nodes in this subtree together with their estimated
values.
%N.B. MCTS should clear some, if not all, of its slate during each iteration.
%In particular, it is important to run new rollouts, even at states visited last iteration, since the value function has changed.
MCTS employs not one, but two, policies as it builds this subtree of
nodes and their estimated values.  It employs the \emph{tree policy\/}
if ever it encounters a node it has visited before (i.e., a node
already in its subtree); and it employs a \emph{rollout policy\/} if
ever it encounters a node it has not visited before (i.e., a node
\emph{not\/} already in its subtree) in order to assess the value
of that node, and then it adds it to its ever-growing subtree/data set.

\begin{table}[h!]
\begin{enumerate}
\item Clean the slate: i.e., start from an empty data set.
    
\item For $M$ simulations:

\begin{enumerate}
\item While the game is not over:
    
\begin{enumerate}
\item If the current node has been visited before:
    
\begin{enumerate}
\item Make a move using the tree policy $\pi$
\end{enumerate}

\item If the current node has \emph{not\/} been visited before:
  
\begin{enumerate}
\item Run $N$ rollouts to depth $d$, returning the average value of all nodes reached at depth $d$

\item Optional, but very common: Walk back up the tree, averaging this value estimate
  for the current node into all its ancestors' value estimates

\item Optional: Update the tree policy using the new information gleaned from the rollouts
\end{enumerate}
\end{enumerate}
\end{enumerate}

\item Create a data set consisting of all nodes visited by the tree policy and their estimates. \\
  (Note that reliable estimates are only produced for nodes visited by the tree policy.)
\end{enumerate}
\caption{MCTS, for use in Step 2(a) of Policy Iteration}
\label{alg:mcts}
\end{table}

The simplest tree policy is an $\epsilon$-greedy policy, which chooses
an optimal action w.r.t.\ the current value function with probability
$1-\epsilon$, and one of the other actions uniformly at random with
total probability $\epsilon$.  The simplest rollout policy generates
only a single action (i.e., $N = d = 1$); that is, the most
straightforward way to estimate a value at a state is to use its
1-step Bellman update (i.e., TD(0)), \emph{\`{a} la\/} TD-Gammon~\cite{tdgammon}.

%%% DELETED, b/c the variance would be very high if we were to use just one TD(1) estimate
%In the usual RL fashion, these new and improved estimates can be 1-step Bellman updates (i.e., TD(0)) or Monte Carlo estimates (i.e., TD(1)), or anything in between.

A more sophisticated, and typical, rollout policy (which gives the
policy its name) is one that simulates the game multiple times at the
current state to (Monte Carlo) estimate that state's value.  In other
words, it runs \emph{inner\/} simulations within the \emph{outer}
MCTS.  The term \emph{rollout\/} is used to describe one of these
inner simulations.  By generating multiple rollouts at a state, using
a policy based on the current value function (e.g.,
$\epsilon$-greedy), and then averaging the rollouts' values, we
produce a TD(1) estimate of the current state's value based on $d$
steps of lookahead, where $d$ is the depth of the rollout.
%instead of just 1 step of lookahead, which we could have obtained using a TD(0) update, as in TD-Gammon.
%
After enough%
\footnote{Say $L$, another hyperparameter.}
learning epochs, it becomes natural to estimate the value at a state
$d$ steps ahead of the current state using the current value function;
but early in the learning process, it is natural to estimate this
value using an heuristic evaluation function (e.g., the Voronoi
heuristic), as was the practice%
\footnote{It is common to speculate that the initial heuristic
  evaluation function, which was arrived at by learning from human
  expert games, was the secret sauce in
  AlphaGo~\cite{silver16}---essentially, seeding policy iteration with
  a smart prior policy.  This hypothesis was debunked, however, when
  AlphaZero~\cite{silver18}, succeeded without a smart
  initialization---from essentially a blank slate.}
in AlphaGo~\cite{silver16}.
%
Finally, the current state's new estimate can be averaged into the value
estimates at all its ancestors in the tree.

An alternative tree policy that works in conjunction with a rollout policy
%beyond a simple $\epsilon$-greedy policy based on the \emph{current\/} value function
would be an $\epsilon$-greedy policy again, but one based on the (new
and improved) estimated values produced by the rollouts, instead of
the current value function.  More sophisticated still would be a tree
policy that trades off between exploration and exploitation using
multi-armed bandit technology: i.e., choose ``arm'' $j$ that maximizes
\[
\bar{x}_j + c \sqrt{\frac{\ln n}{n_j}} \enspace ,
\]
where $\bar{x}_j$ is the average value of arm $j$, $n_j$ is the number
of times arm $j$ has been selected, $n$ is the number of times the
current state has been visited, and $c$ is a constant, often
$\sqrt{2}$.  The exploitation side of this policy is guided by the
first term, $\bar{x}_j$; arms with higher values are more likely to be
selected.  The exploration side is guided by the second term, which
represents the width of the confidence interval associated with the
estimate $\bar{x}_j$.  Arms that have not been explored sufficiently
have wider confidence intervals.  This tree policy favors exploring
these arms until their confidence intervals shrink.

In summary, most implementations of policy iteration run some variant
of MCTS in Step 2(a), thereby simulating playing the game, making
moves at each state based on the current value function, all the while
collecting data (i.e., new and improved estimated values) at the
states visited during these simulations.  In the usual RL fashion,
these new and improved estimates can be 1-step Bellman updates (i.e.,
TD(0)),
% generated without a rollout policy
or Monte Carlo estimates (i.e., TD(1)),
% generated with a rollout policy
or anything in between.

\if 0
The rollout policy is employed at nodes which have not yet been
visited (or evaluated).  As a node cannot be inserted into the subtree
(in any meaningful way) without an associated value, multiple
simulations are run---so-called rollouts---to estimate its value.  A
\emph{rollout\/} is a game trajectory that proceeds from the current
node to either the end of the game or the maximum rollout depth,
whichever is shorter.  At that point the value of the node is
assessed, either by observing the winner and loser, in the former
case, or by applying an heuristic evaluation function (e.g., Voronoi)
to predict the winner and loser, in the latter.  The average value of
these rollouts is then taken as the initial estimate of the node's
value, so that it can be inserted into the subtree.  This value is
also ``backed-up'' to the root of the tree, meaning it is averaged
into the estimated values of all the node's ancestors, and all of
their visit counts are updated accordingly.

It is probably wise to inject a little bit of noise into even the
smartest of heuristic rollout policies---if your heuristic were
perfect, you would have quit after solving the game using traditional
adversarial search algorithms, without ever having ventured into the
world of reinforcement learning.
\fi

\if 0
Like all MCTS algorithms, AlphaGo simulates the game multiple times in
order to build a subtree in which it stores the data from which it
learns.  During these simulations, it employs a tree policy to choose
its actions.  This tree policy balances exploration and exploitation,
using something like an $\epsilon$-greedy policy, which chooses an
optimal action with probability $1-\epsilon$, and all other actions
with probability totaling $\epsilon$.
%
But before making this choice, AlphaGo simulates multiple game
trajectories at the current node as usual---until either the game ends
and the value of the rollout is known with certainty, or until the
maximum rollout depth is encountered, at which point a heuristic
evaluation function is applied.  It then backs up the value obtained,
incorporating it into the value estimates all along the rollout until
it reaches the current node.  In this way, AlphaGo uses rollouts to
perfect its tree policy.  Consistent with other MCTS algorithms a
possible rollout policy for AlphaGo is the aforementioned multi-armed
bandit strategy, and a possible heuristic evaluation function is the
Voronoi heuristic.
\fi

Adversarial search algorithms make the assumption that the opponent is
a rational agent.  RL for game-playing, in contrast, attempts to learn
a best response to a given opponent strategy---rational or otherwise.
This strategy is encoded in the transition probabilities of an MDP,
and thus is invoked during the data-generation step in policy
iteration.  In particular, whenever a tree policy or a rollout policy
is simulated, the next state in the MDP is a node two levels down in
the adversarial game tree, which obtains first by following the
learner's tree or rollout policy, and second by simulating the
behavior of the opponent.  In symmetric games like Tron, an
alternative to learning given an opponent' strategy is \emph{learning
  in self-play}.  In this setup, the rollout policy still simulates
both players actions, but it uses a recent%
\footnote{For example, the most recent.  It does not use the current
  policy to encourage stability, and ultimately, convergence.}
policy.  As for the tree policy, after it takes its action, the
learner can assume the role of the opponent.  Learning in self-play
can generate more robust strategies than learning a best response to a
specific opponent's strategy.


\section{Code}
We have taken the liberty of implementing Tron-141 for you.  We have
implemented it as a derivative of \texttt{AdversarialSearchProblem}, from
Assignment 2: i.e., \texttt{TronProblem} inherits from the abstract class
\texttt{AdversarialSearchProblem}.
Moreover, \texttt{TronState} inherits from \texttt{GameState}.
Thus, your only task is to implement an AI bot to play the game.


\subsection{Code to Modify}
The code you should modify can all be found in two files:

\begin{itemize}
\item \texttt{bots.py} contains stencil code for the \texttt{StudentBot} class,
  where your main task is to complete the \texttt{decide} function,
  which indicates the move your \texttt{StudentBot} decides to take.

  This module also contains code for \texttt{RandBot} and
  \texttt{WallBot}, two bots against which you can test your
  \texttt{StudentBot}.  You can also write your own baseline bots in
  this file to test your bot against.
  
  \textbf{Hint}: We recommend you read through the code for the bots
  we have already implemented.  Doing so will help familiarize with
  the different members of \texttt{TronProblem} and \texttt{TronState}.

\item \texttt{support.py} contains a function called
  \texttt{determine\_bot\_functions} to which you can add clauses that
  correspond to new bots you write in \texttt{bots.py}.  This is only
  necessary if you create a bot other than \texttt{StudentBot} for the
  purpose of testing \texttt{StudentBot}.
\end{itemize}

\noindent
\textbf{Note:} You may \textit{not\/} use the \texttt{signal} library
or catch \texttt{support.TimeoutException} as part of your solution.


\subsection{Code \emph{not\/} to Modify}
\textbf{Note:} Do not modify any of these stencil files.

\begin{itemize}
\item \texttt{tronproblem.py} defines the \texttt{TronProblem} and \texttt{TronState} classes.
%
  The function in this file that we expect to be most useful to you
  is the static method \texttt{get\_safe\_actions(board, loc)}, which
  returns the set of actions a player can take from the position
  \texttt{loc} that would not result in a collision.

\item \texttt{gamerunner.py} runs the game.  We describe some of its
  command-line arguments in the next section.

\item \texttt{trontypes.py} contains constants that are used to
  identify cells on the board and types of powerups.

\item \texttt{boardprinter.py} handles printing the board and game
  information to the terminal.

\item \texttt{adversarialsearchproblem.py} is identical to the file of
  the same name we distributed with the Adversarial Search assignment.
  The \texttt{TronProblem} class inherits from the
  \texttt{AdversarialSearchProblem} class, and \texttt{TronState}
  inherits from \texttt{GameState}.
\end{itemize}


\subsection{Testing your Solution}
As you develop your bot, you can evaluate it by playing matches
against various opponents on a variety of maps.  Specifically, we are
releasing four TA bots and eight sample maps.
%
You can test your \texttt{StudentBot} in simulated games against other
bots using the \texttt{main} function in \texttt{gamerunner.py}.  This
function takes a few command line arguments, the most important of
which are:

\begin{itemize}
\item \texttt{-bots} lets you specify which bots to play against one another. The syntax is \texttt{-bots <bot1> <bot2>}

\item \texttt{-map} lets you select the map that the game is to be played on. The syntax is \texttt{-map <path to map>}

\item \texttt{-multi\_test} lets you run the same game setup (choice of bots and map) multiple times.
  You may want to run multiple tests with the \texttt{-no\_image} flag, so the games are played more quickly.
  (Printing to the terminal slows things down.)
  To do so, use \texttt{-multi\_test <number of games> -no\_image}.

\item \texttt{-no\_color} runs the game without coloring the board printout.
  You should use this option if coloring causes display issues.
\end{itemize}

For example, you can test your \texttt{StudentBot} against \texttt{RandBot} on the \texttt{joust} map using \\
\texttt{python gamerunner.py -bots student random -map maps/joust.txt}

You can test your \texttt{StudentBot} against \texttt{WallBot} 100 times with no visualizer on the \texttt{empty\_room} map with
\texttt{python gamerunner.py -bots student wall -map maps/empty\_room.txt -multi\_test 100 -no\_image}

\textbf{Note}: When running multiple tests, your bot will move first in every other match.


\paragraph{Opponents}
There are four sample TA bots:

\begin{enumerate}
\item \texttt{RandBot} chooses uniformly at random among all actions that do not immediately lead to a loss.

\item \texttt{WallBot} hugs walls and barriers to use space efficiently.

\item \texttt{TA-Bot1} and \texttt{TA-Bot2} Two more sophisticated TA bots, with secret implementations.
\end{enumerate}

As already mentioned, you can find the code for \texttt{RandBot} and
\texttt{WallBot} in \verb|bots.py|.  The implementation of the other
TA bots is not exposed.  Instead, it is included as a compiled module,
\verb|ta_bots.so|.  You can still test your bot against these bots:
when running \texttt{gamerunner.py}, use the \texttt{-bots} flag with
\texttt{ta1} or \texttt{ta2} as an argument.

In addition to these bots, you should save versions of your own bot as
you work to improve it.  Earlier versions can serve as baselines
against which you can test later versions, to be sure that your
strategy is indeed improving.

%%% TOO RISKY
%You are also free to test your bot against other students' bots, as long as you do not copy each other's code. \amy{seems risky!}


\paragraph{Maps}
There are eight sample maps, available in the \texttt{maps} directory.
Two are empty maps, one big (13x13) and one small (7x7).  You should
use this small map for testing purposes; and you should feel free to
create and test your code on other perhaps even smaller maps as well.
There are two other big maps without powerups, and four with powerups.
As noted previously, you need not tailor your bot to powerups.

All maps are stored in \texttt{.txt} files, using the same characters
that appear in the board printout.  The only exception is the
\texttt{?} character, which represents powerups in the files.  When
\texttt{gamerunner.py} reads in the map files, each \texttt{?} is
replaced by one of the four powerups, chosen uniformly at random
with replacement.


\section{Writeup}
You and your partner (if you have one) should hand in a final writeup
by \textbf{Monday, December 14 at 11:59pm ET}.  In short, this writeup
should describe your Tron-141 bot.

This project is very open-ended.  There are numerous approaches you
might try, only a few of which you can be expected to get working
within the allotted time frame.  Your writeup should include:

\begin{itemize}
\item The back story: What did you try first?  What worked?  What
  didn't work?  How did you eventually arrive at your bot's present
  design?
  
\item A description of how your bot works, sufficiently detailed so
  that the reader could replicate your bot.

\item A description of your bot's known shortcomings, including how
  you would attempt to ameliorate them with more time.
\end{itemize}


\section{Warmup}
For this project, you and your partner (if you have one) will be
assigned a mentor TA to bounce ideas off of.  To make sure you get
started thinking about this project immediately, you are required to
meet with this TA by \textbf{Monday, December 7th at 5:59pm ET}.

To prepare for this meeting, you should familiarize yourself with the
rules of Tron-141, and with the support code.  You should understand
how to implement a basic bot (e.g., one that makes a random move), and
you should also have some ideas about how you are going to build a
more sophisticated strategy.

Also by \textbf{Monday, December 7th at 5:59pm ET}, you and your
partner should turn in an implementation of a basic bot, along with a
short report describing your plans for a more sophisticated strategy.
The contents of this report should form the basis of your meeting with
your mentor TA.


\section{Tournament}
We will be running a daily Tron-141 tournament, beginning on
\textbf{Monday, December 7th at 11:59pm ET}, so you can see how your
bot stacks up against other students' bots.  To submit your bot to the
tournament:

\begin{enumerate}
\item Copy your code from \texttt{~/course/cs1410/Tron/} to
  \texttt{$\sim$/course/cs1410/TronTournament/}.

\item Add a custom bot name,
  by adding the following to your \texttt{StudentBot.\_\_init\_\_} function:
    
  \hspace*{6mm} \texttt{self.BOT\_NAME = "My custom bot name"}

  Be creative!

\item Submit your tournament bot using \texttt{cs1410\_handin TronTournament}.
\end{enumerate}

Please submit only one bot per group!  The tournament will grab the
latest submissions every night around midnight, run a tournament, and
then post the results online
\href{http://cs.brown.edu/courses/csci1410/tron_results.html}{here}.
%
%Participation in this tournament is not required but is strongly encouraged.
The tournament winner will be showered in praise, and might even be awarded a cash prize!


\section{Grading}
Your bot should be able to defeat \texttt{RandBot} virtually all of
the time, \texttt{WallBot} virtually all of the time on most maps, and
\texttt{TA-Bot1} and \texttt{TA-Bot2} most of the time---all on boards
without powerups.  We will test your bots on the \texttt{empty\_room},
\texttt{center\_block}, and \texttt{diagonal\_blocks} maps, as well as
on at least one secret map.  If you indicate in your writeup that your
strategy is designed to handle powerups, then the secret map(s) will
include powerups.  See \texttt{rubric.txt} for more details.


\section{Capstone}
These are the additional requirements for those taking this course as a capstone:

\begin{enumerate}
\item You must work independently.

\item Your bot \emph{must\/} incorporate machine learning.

\item Your writeup must include an evaluation of your machine learning
  approaches, including experiments demonstrating which approaches
  worked and why.  While you should of course include plots optimizing
  your hyperparameters, you should also compare different algorithmic
  approaches---for example, by comparing the strategies they learn on
  small maps, after their hyperparameters have been optimized.
\end{enumerate} 


\section{A Note About TA Hours}
As already mentioned, this project is very open-ended.  There are many
viable solutions, and it is not obvious \emph{a priori\/} what will
work and what will not.  As such, you should not come to TA hours
expecting definitive ``Yes, this will work'' or ``No, that definitely
won't work'' kinds of answers.  Instead, you should view TA hours for
this project as an opportunity to talk through your ideas to get a
second (or third) opinion.  You can also ask the TAs to review past
course material with you as necessary.


\section{Install and Handin}
To install, run \verb|cs1410_install Tron| in \verb|~/course/cs1410|.

To hand in your warm-up code,
copy your project from \verb|~/course/cs1410/Tron/| 

\noindent to \verb|~/course/cs1410/TronWarmup/|.
Then run \verb|cs1410_handin TronWarmup| in

\noindent \verb|~/course/cs1410/TronWarmup|.
In addition, please submit the written portion of the Warm-up via Gradescope.

To hand in your final code, run \verb|cs1410_handin Tron| in \verb|~/course/cs1410/Tron|.
In addition, please submit the written portion of the final project via Gradescope.


\textbf{Please submit only one warm-up and final project per group!}
When submitting on Gradescope, you must specify all members of your
group.  Note that you can do this after you upload your document and
hit ``Submit.''  Use the ``View or edit group'' option at the top
right of the page.

Finally, please note that since this is a final project,
the normal resubmission policy does not apply.
You may not use any late days.
\textbf{December 14 is the hard deadline for all parts of the project.}
Do not wait until the last minute to submit.
Submit on Sunday, and update on Monday as necessary.

In accordance with the course \href{https://forms.gle/DqfbBY8jdaqenRoa9}{grading policy}, your
written homework should not contain your name, Banner~ID, CS~login, or any other personally
identifiable information.


\nocite{lanctot13}
\nocite{teuling12}
\nocite{samothrakis10}

\bibliographystyle{plain}
\bibliography{handout}


\appendix

\section{Powerups}
\label{app:powerups}
Below are two example 13x13 game boards \emph{with powerups}.  The one
on the left is the initial board, and the one on the right is the same
board after Player 1 has moved down and Player 2 has moved up.

\begin{verbatim}
             ###############               ###############
             #1            #               #x            #
             #             #               #1            #
             #       *     #               #       *     #
             #             #               #             #
             #    @     ^  #               #    @     ^  #
             #             #               #             #
             #        !    #               #        !   2#
             #            2#               #            x#
             ###############               ###############
\end{verbatim}

The \verb|*|, \verb|@|, \verb|^|, and \verb|!| symbols represent
powerups (traps, armor, speed, and bombs, respectively), which
players obtain by moving into a cell that contains one.
Powerups are not an essential aspect of Tron;
they were invented by past CSCI 1410 TAs.
%
A player obtains a powerup moving into a cell that contains one.
There are four different types of powerups.

\if 0
\begin{enumerate}
\item Trap: Represented by \texttt{*} on the map.

\item Armor: Represented by \texttt{@} on the map.

\item Speed: Represented by \texttt{\^} on the map.

\item Bomb: Represented by \texttt{!} on the map.
\end{enumerate}
\fi

\textbf{Trap} powerups create up to three new barriers on the border
of the 5x5 area surrounding the opponent.  The -'s on the board below
denote the locations at which barriers could be placed near Player 1,
if Player 2 moves into a cell with a trap powerup.  The new barriers'
locations are selected uniformly at random among the unoccupied cells
on this square.  (If fewer than three cells on this square are
unoccuppied, fewer than three new barriers are created.)

\begin{verbatim}
                           ###############
                           #             #
                           #    -----    #
                           #    -   -    #
                           #    - 1 -    #
                           #    -   -    #
                           #    -----    #
                           #             #
                           #           2 #
                           ###############
\end{verbatim}

\textbf{Armor} powerups allow a player to travel through a single
barrier.  After a player obtains an armor powerup, it is applied
automatically, if ever the player moves into a cell with a barrier.
Note that an armor powerup only allows players to travel through
\textit{barriers} (represented on the map by \texttt{x}'s), not
through permanent walls (\texttt{\#}) or other players.
%\amy{this seems like a silly restriction. why not just let them move into a permanent wall? they will lose on the next turn regardless, so why not? just in case it is a very close game, in which case the powerup which buys them one move cannot save them? i think it \emph{should\/} be able to save them in that case.}

\textbf{Speed} powerups are like a speed boost.  They afford a player
four consecutive, \emph{mandatory\/} moves.

\textbf{Bomb} powerups destroy all the barriers (\texttt{x}'s) in the
9x9 area surrounding the bomb, replacing them with open space.  They
are activated immediately when a player moves into a cell that
contains one.  The \texttt{-}'s on the board below denote the
locations where barriers would be destroyed if the bomb in the center
exploded.

\begin{centering}
\begin{verbatim}
                           #################
                           #               #
                           #               #
                           #   ---------   #
                           #   ---------   #
                           #   ---------   #
                           #   ---------   #
                           #   ----!----   #
                           #   ---------   #
                           #   ---------   #
                           #   ---------   #
                           #   ---------   #
                           #               #
                           #               #
                           #################
\end{verbatim}
\end{centering}

\end{document}

