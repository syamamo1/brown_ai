\documentclass{article}
\usepackage{amsmath}
\usepackage[utf8]{inputenc}
\usepackage{hyperref}
\hypersetup{
    colorlinks=true,
    linkcolor=blue,
    filecolor=magenta,
    urlcolor=blue,
}
\usepackage{fullpage}
\newcommand{\amy}[1]{{\color{blue}[Amy: #1]}}
\newcommand{\ella}[1]{[Ella: #1]}

\title{CSCI1410 Fall 2020 \\
Assignment 1: Search}

\date{Code Due Monday, September 28 at 11:59pm ET\\ [1ex]
Writeup Due Tuesday, September 29 at 11:59pm ET}

\begin{document}

\maketitle


\section{Goals}

In this assignment, you will implement depth-first,
iterative-deepening, A*, and iterative-deepening A* search,
and you will use those search algorithms to solve instances of a Tile Game.


\section{Introduction}

The Sliding Tiles puzzle is a famous Tile Game.
It is played on an $n$-by-$n$ grid ($n$ is usually 3 or 4, or perhaps 5),
with numbers occupying all but 1 of the $n^2$ cells,
and an empty space occupying the last.
Starting from a state with the numbers arranged at random,
the goal is to rearrange all the tiles into ascending order,
with the space in the bottom right-hand corner (as shown).
The name derives from the fact that the tiles can slide over the empty space.

The Tile Game for this assignment is a variant on Sliding Tiles,
also played on a $n$-by-$n$ grid, with numbers occupying the cells.
There is no space, however; all $n^2$ tiles are occupied.
The game is played by swapping any two adjacent tiles,
where adjacency is defined grid-wise (diagonal tiles are not adjacent) and does not wrap around.
Like in the original Sliding Tiles puzzle,
the goal is to arrange the numbers in ascending order.

The goal state on a 3-by-3 board is:
\[
\begin{bmatrix}
1 & 2 & 3 \\
4 & 5 & 6 \\
7 & 8 & 9
\end{bmatrix}
\]

Here is one way to arrive at this goal, from a random start state: \\

\begin{tabular}{ccccccc}
$\begin{bmatrix}
6 & 1 & 2 \\
7 & 8 & 3 \\
5 & 4 & 9
\end{bmatrix}$
&
$\begin{bmatrix}
1 & 6 & 2 \\
7 & 8 & 3 \\
5 & 4 & 9
\end{bmatrix}$
&
$\begin{bmatrix}
1 & 2 & 6 \\
7 & 8 & 3 \\
5 & 4 & 9
\end{bmatrix}$
&
$\begin{bmatrix}
1 & 2 & 3 \\
7 & 8 & 6 \\
5 & 4 & 9
\end{bmatrix}$
&  
$\begin{bmatrix}
1 & 2 & 3 \\
7 & 8 & 6 \\
4 & 5 & 9
\end{bmatrix}$
&  
$\begin{bmatrix}
1 & 2 & 3 \\
4 & 8 & 6 \\
7 & 5 & 9
\end{bmatrix}$
&
$\begin{bmatrix}
1 & 2 & 3 \\
4 & 5 & 6 \\
7 & 8 & 9
\end{bmatrix}$ \\ [5ex]
\end{tabular}

The search space (i.e., the number of states) in a Tile Game is
exponential, because there are $m! \in \Omega(m^m)$ ways to order $m$
consecutive integers.  As we cannot possibly enumerate this space for
all but the smallest choices of $m$, you will solve instances of the
Tile Game using heuristic search.  To do so, you will implement both
blind and informed search algorithms.  The informed algorithms rely on
heuristics, which you will develop by encoding domain-specific
knowledge.  By their very nature, the informed algorithms are more
sophisticated than the blind algorithms.  Indeed, if your heuristics
are well designed, you will find that you can solve larger Tile Game
instances using informed search algorithms.

The informed search algorithms that you will implement are variants on
A* search.  A* is optimal when it is endowed with an admissible (i.e.,
optimistic) heuristic.  When its heuristic is inadmissible, optimality
is no longer guaranteed.  Still, you will experiment with inadmissible
heuristics in this assignment.  Since inadmissible heuristics are less
constrained than admissible ones, they are often faster, so when they
are not very inadmissible (i.e., when they are only ever slightly
pessimistic), they may find optimal or only slightly suboptimal
solutions faster than admissible heuristics.

In Modern AI, where the goal is usually to design agents that make
rational decisions relative to some objective, the name of the game is
to develop heuristics that trade off between optimality and complexity
(in terms of time and/or space).  That is precisely the subject matter
of this assignment.


\section{Data Structures}

In this (and perhaps all CSCI1410) assignment(s), we worked out the
problem of how to represent the data---in this case, a Tile Game---for
you.  We chose to represent a state in the Tile Game as an $n$-tuple
of $n$-tuples.  The outer $n$-tuple represents the rows in a Tile
Game, while the inner $n$-tuples represent the entries in the
corresponding row's columns.  For example, the goal state is
represented as:
\[    
((1,2,3),(4,5,6),(7,8,9))
\]


\section{Algorithms}

Once again, the algorithms you will implement in this assignment are
depth-first (DFS),
iterative-deepening (IDS), A*, and iterative-deepening A* (IDA*) search.
These algorithms are described in Sections 3.3 (generic search),
3.4 (blind search), and 3.5 (informed search) of Russell and Norvig.

Recall from class that search can be conceptualized as expanding a tree.
Note, however, that no search algorithm builds this search tree in its totality.
On the contrary, these algorithms simply move through the search space,
by taking actions that lead them from one state to another.

Moreover, while it is convenient to think about searching a tree, a
search space is not necessarily a tree.  In general, \emph{it is a
  graph}, because it is possible to revisit states.  In the Sliding
Tiles puzzle, for example, you can move a tile into the empty space
and back again indefinitely.  Hence, to the extent that memory
permits, your implementation should keep track of previously visited
states, and should not visit them again.

\section{Your Tasks}

Your primary task is to implement more and more sophisticated
algorithms so as to solve larger and larger instances of the Tile
Game.  After implementing these algorithms, you will run a series of
experiments to test them on games of varying size, and you will
summarize your findings.  You might find, for example, that DFS can
only solve Tile Games of size 2, while IDA* with an admissible
heuristic can optimally solve Tile Games of size 3, and with an
inadmissible heuristic, half the Tile Games you test it on of size 4.


\subsection{Coding}

In \verb|search.py|, you will implement depth-first (\verb|dfs|), iterative deepening (\verb|ids|),
A* (\verb|astar|), and IDA*(\verb|id_astar|).
Each search function takes in a \verb|SearchProblem|, called \verb|problem|,
and outputs a \textit{path\/} through the search tree.
This path should be represented as a list of states, where the first
element is the start state and the last element is a goal state.

To complete your implementation of the informed search algorithms,
you should use your knowledge of the Tile Game to develop
admissible and inadmissible heuristics.
You are welcome to experiment with multiple heuristics,
but you should submit only one of each,
in \verb|admissible| and \verb|inadmissible| as appropriate.
Section 3.6 of the textbook provides examples of heuristic functions.

With the exception of your heuristics, which exploit domain knowledge,
your implementations of the search algorithms should be problem agnostic,
meaning they should work on any \verb|SearchProblem|:
e.g., Sliding Tiles, as well as the Tile Game,
and even a routing problem like finding a driving route from Providence
to Boston or a flight from Providence to Tokyo.


\subsubsection*{Tips}

\begin{itemize}

\item
  Don't forget to write your tests \emph{before\/} you write any code!
  Otherwise, how would you know whether your implementation was correct?
  Never, and we mean \emph{never\/}, write your tests based on the output of your implementation.

\item
  Don't copy code.
  (And we don't mean from other people; \emph{obviously}, you should not do that!)
  Don't copy code from yourself.
  If you find some portion of your code has multiple uses,
  abstract it out, so that you can call it from multiple places.

\end{itemize}


\subsection{Writeup}

In addition to your code, you should also submit a pdf in which you
answer the following questions, one at a time.  While not required, we
recommend that you use \LaTeX to typeset your work.

\begin{enumerate}

\item Describe your admissible heuristic, and argue that it is admissible.
  Explain why you chose this heuristic.

\item Describe your inadmissible heuristic. Give examples where it is
  inadmissible.  Informally speaking, how inadmissible is it?
  Explain why you chose this heuristic.

\item Design and run experiments to test the various algorithms on
  problem instances of varying size (say, 10 problems of each size).
  Test the blind algorithms on smaller problem instances, and the
  informed ones on larger problem instances.  Report your findings in
  a table, and summarize them in a few sentences.

  Your table/summary should answer questions like: How often do the
  blind algorithms succeed as compared to the informed algorithms, the
  latter with both admissible and inadmissible heuristics?  On problem
  instances where A* and/or IDA* succeeds, so that you know the length
  of the optimal solution, how suboptimal, on average, are the paths
  that the not-provably-optimal algorithms find (i.e., what is the
  average difference between the length of the optimal path and the
  suboptimal ones?).

\end{enumerate}

\subsection{Ethics Questions}
Read \href{https://www.usnews.com/news/national-news/articles/2018-05-07/why-some-cities-have-had-enough-of-waze}{this} article about Google's popular GPS navigation software Waze and answer the following questions. Together, the answers should fill half of a page. Include you answers on the same document as your writeup.

\begin{enumerate}

\item As we saw in lecture, Google uses the A* algorithm in its mapping software for finding the shortest path. By defaulting to the shortest path, Waze directly prioritizes the needs of its users. When developing a product, whose needs should be prioritized? Do you believe Waze, and other tech companies, are obligated to always be “user first”? Consider the impacts of a technology company like Waze putting their users' needs second and instead prioritizing the community they operate in.

\item According to the article, the company's stance is the following:
"It's important to note that Waze does not 'control' traffic but our maps do reflect public roads that federal and local authorities have identified and built for its citizens. If the city identifies a dangerous condition, it is their responsibility to legally reclassify a road, which will then be reflected on the Waze map." When a company's technology exacerbates longstanding public problems, with whom does the responsibility lie to address these issues? Do you agree with Waze's stance that it is the city’s responsibility?
\end{enumerate}


\noindent
Please submit your writeup for both 5.2 and 5.3 via Gradescope (see the Gradescope guide on the course website for more details).


\section{The Code Files}

\subsection{Files to Modify}

\begin{itemize}

\item \verb|search.py| - This is where you will implement your search algorithms.

\end{itemize}


\subsection{Core Source Code}

\begin{itemize}

\item \verb|searchproblem.py| - This file contains an abstract class, \verb|SearchProblem|,
  that can represent search problems.
  The \verb|SearchProblem| class has three abstract methods that are shared among all search problems.
  Be sure to read the function headers and their docstrings before you begin.

\item \verb|tilegameproblem.py| - This file contains the \verb|TileGame| class, which extends the \verb|SearchProblem| class.
  It contains implementations of the abstract methods in \verb|SearchProblem|.
  It also contains some helper functions that you may find useful when you code your heuristics.
   
  You will notice that the dimension of the Tile Game is adjustable.
  You should take advantage of this feature: while developing your
  code, you can test it on 2-by-2 games, which should run more quickly
  than 3-by-3 games or 4-by-4 games.  But once you are confident in
  your implementations, you should of course test them on larger
  problem instances as well.

\end{itemize}


\subsection{Testing Source Code}

\begin{itemize}

\item \verb|dgraph.py| - This file contains an implementation of a \verb|DGraph|,
  which represents directed graphs, as a \verb|SearchProblem|.
  We will use \verb|DGraph| to test your implementations.
  You should use \verb|DGraph| as well for the same purpose: to design test cases for your searches.
  
  \verb|DGraph| represents directed graphs using matrices.
  Each state is each represented by a unique index in $\{ 0, 1, ..., S-1 \}$,
  where $S$ is the number of states,
  and the cost of moving directly from state $i$ to state $j$ is the entry in the matrix at row $i$, column $j$.
  If it is impossible to move directly from $i$ to $j$ (i.e., $j$ is not a successor of $i$),
  then the entry in the matrix at row $i$, column $j$ is set to \verb|None|.

\item \verb|unit_tests.py| - This file contains a testing suite with some trivial test cases.
  To run these tests, execute \verb|python unit_tests.py| inside the virtual environment.
  We encourage you to write additional unit tests, as well as more sophisticated tests
  (integration, end-to-end, etc.).

\end{itemize}


\subsection{Support Code}

The queue module contains \texttt{Queue}, \texttt{LifoQueue}, and \texttt{PriorityQueue},
which you are free to use to implement your search algorithms.
All three implement the \texttt{put}, \texttt{get}, and \texttt{empty} functions.
To add an item to a \texttt{PriorityQueue} with a given priority, add a tuple in the form of \textit{(priority, item)}.
By default, \texttt{PriorityQueue} retrieves the tuple with the \textit{lowest} priority value.
Visit \url{https://docs.python.org/3/library/queue.html} for more information about the module.


\section{Grading}

We will give you your score based on the rubric in \verb|rubric.txt|.
Here are some details about the rubric:

\begin{itemize}
\item We will check each of your search algorithms to ensure that states are expanded in a proper order.
The autograder considers a state to be expanded whenever \verb|get_successors| is called on it.
So be sure that \verb|get_successors| is not called unnecesarily.
For most search problems, there will be many correct orders of expanding the states for each search algorithm.
Our grading scripts will give you full points if you expand each search problem in any of the proper orders.

\item For your \textbf{admissible} Tile Game heuristic,
  you will be graded based on how many states A* expands.
  Your score for this part will be determined by the following formula:

\begin{equation}
10 \cdot \frac{n_{\rm{ours}}}{n_{\rm{yours}}} \enspace ,
\end{equation}

where $n_{\rm{ours}}$ and $n_{\rm{yours}}$ are the number of nodes
expanded by our heuristic and your heuristic, respectively, on a
pre-selected suite of Tile Game instances.

You can score your heuristic on your own on a department machine by using the following command inside the virtual environment:
\[
\verb|cs1410_test_heuristic /path/to/your/search.py|. 
\]

\item Your \textbf{inadmissible} heuristic will not be autograded,
  but will be graded in your writeup. 

\end{itemize}


\section{Virtual Environment}
Your code will be tested inside a virtual environment that you can activate using 
\[
\verb|source /course/cs1410/venv/bin/activate|. 
\]
Read more about it in the course grading policy document \href{https://cs.brown.edu/courses/csci1410/documents/grading_policy.pdf}{here}.


\section{Install and Handin Instructions}
To install, run \verb|cs1410_install Search| in \verb|~/course/cs1410/|.

To handin, run \verb|cs1410_handin Search| in \verb|~/course/cs1410/Search/|,
which should contain your \verb|search.py| file.

To handin, run
\verb|cs1410_handin ASSIGNMENT_NAME| in \verb|~/course/cs1410/ASSIGNMENT_NAME/|.

IMPORTANT NOTE: In addition, please submit an online \href{https://forms.gle/DqfbBY8jdaqenRoa9}{collaboration policy}
and the written portion of the assignment to Gradescope (instructions
available on Piazza).  Since we cannot grade your work until you
submit a signed collaboration policy, you should submit your signed collaboration policy,
even if you plan to submit your writeup later.

In accordance with the course \href {http://cs.brown.edu/courses/csci1410/documents/grading_policy.pdf}{grading policy}, your written homework should
not have any identifiable information on it, including banner ID, name, or cslogin.

\end{document}
